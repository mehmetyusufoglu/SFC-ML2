\documentclass[12pt,notitlepage]{article}
\usepackage[utf8]{inputenc}
\usepackage[margin=1in]{geometry}
\usepackage{setspace}
\usepackage{amsmath}
\usepackage{hyperref}

\begin{document}
\begin{center}
{\Large \textbf{Heterodox Development Policy Ordering and Composition: Findings from an Open-Economy SFC Model}}\\[0.75em]
\textit{Mehmet Yusufoglu}\\
\textit{\href{https://www.hzdr.de/db/Cms?pOid=57317&pNid=1227}{HZDR/Casus}}\\
\textit{m.yusufoglu@hzdr.de, \textperiodcentered mehmetyusufoglu01@gmail.com}
\end{center}

\vspace{1em}

\begin{abstract}
\doublespacing

Heterodox development strategies for semi-peripheral economies typically combine expansionary public investment, credit socialization, and capital controls. We investigate whether the \textbf{sequencing} of these policies and focusing on some specific aims at public investment matters. Specifically, we test whether targeting public investment toward \textbf{cost-reducing infrastructure} (transport, energy, social housing) \textbf{before} credit reforms creates more favorable transition conditions than simultaneous implementation.

Our hypothesis: infrastructure investments lower reproduction costs for workers and input costs for firms, generating profit buffers that prevent capital flight when financial institutions are later transformed. Attempting credit socialization without prior cost reductions triggers balance-of-payments crises. We test this with an open-economy Stock-Flow Consistent (SFC) model calibrated to Turkish data (2010--2020), comparing three scenarios: (1) baseline capitalism, (2) infrastructure without institutional change, and (3) infrastructure followed by credit socialization plus capital controls.

\href{https://github.com/mehmetyusufoglu/SFC-ML2/tree/main}{Simulation} and 500-run Monte Carlo experiments confirm the sequencing hypothesis. Infrastructure alone (Scenario 2) lowers costs by 19\% but explodes external debt—consistent with Turkey's recent difficulties. Sequencing infrastructure \textbf{before} credit socialization (Scenario 3) stabilizes the transition: profit rates stay above global benchmarks, capital flight falls to zero, and the economy sustains high growth with controlled inflation. The mechanism is the material buffer generated by infrastructure, not institutional change alone. China's 1980--1993 infrastructure surge preceding the 1994 banking reform suggests this pattern may generalize.

While focused on infrastructure-credit sequencing, the model supports investigation of additional policy dimensions: profit-rate targeting mechanisms, interest-rate policies by development versus private banks, exchange-rate management strategies, and the balance between cost-reducing versus demand-stimulating expenditure. The sequencing question examined here represents one slice of a richer policy space for future work.

	\textbf{Implication:} Progressive transitions may benefit from sequencing cost-reducing infrastructure before credit socialization to limit capital flight. The paper also reports transition effects on employment, productivity, and external balances to show how ordering stabilizes the path. The contribution is a parsimonious SFC formalization illustrating why reform order matters within a broader heterodox policy space.

\end{abstract}

\noindent \textbf{Keywords:} Stock-flow consistent modeling, semi-peripheral economies, transitional dynamics, capital flight, crisis mechanisms, Turkey

\noindent \textbf{JEL Classifications:} E12, E17, E61, F32, O11, O23, P51

\section*{References (Selected)}

\noindent Godley, W. and Lavoie, M. (2007). \textit{Monetary Economics: An Integrated Approach to Credit, Money, Income, Production and Wealth}. Palgrave Macmillan.

\noindent Orhangazi, Ö. (2019). \textit{Financialization and the Turkish Economy}. Palgrave Macmillan.

\noindent Reyes, L. (2025). ``Analysis of Alternative Economic Policies for France: A Simulation-Based Approach (2025--2035).'' Paper presented at Cassino SFC Conference.

\noindent Storm, S. (2023). ``The Case for Inducing Productivity Growth.'' \textit{Industrial and Corporate Change} 32(4): 891--910.

\vspace{1em}
\noindent \textbf{Main SFC code repository:} \url{https://github.com/mehmetyusufoglu/SFC-ML2/tree/main}

\vspace{0.5em}
\noindent \textit{Word count: 280 words}

\end{document}
